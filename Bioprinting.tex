%%%%%%%%%%%%%%%%%%%%%%%%%%%%%%%%%%%%%%%%%
% Simple Sectioned Essay Template
% LaTeX Template
%
% This template has been downloaded from:
% http://www.latextemplates.com
%
%
%%%%%%%%%%%%%%%%%%%%%%%%%%%%%%%%%%%%%%%%%

%----------------------------------------------------------------------------------------
%	PACKAGES AND OTHER DOCUMENT CONFIGURATIONS
%----------------------------------------------------------------------------------------

\documentclass[12pt]{article} % Default font size is 12pt, it can be changed here
\usepackage[utf8]{inputenc}
\usepackage{geometry} % Required to change the page size to A4

\geometry{a4paper} % Set the page size to be A4 as opposed to the default US Letter
\linespread{1.2} % Line spacing

\begin{document}

%----------------------------------------------------------------------------------------
%	TITLE PAGE
%----------------------------------------------------------------------------------------

\begin{titlepage}

\newcommand{\HRule}{\rule{\linewidth}{0.5mm}} % Defines a new command for the horizontal lines, change thickness here

\center % Center everything on the page

\textsc{\LARGE Politecnico di Milano}\\[1.5cm] % Name of your university/college
\textsc{\large Computer ethics}\\[0.5cm] % Minor heading such as course title

\HRule \\[0.4cm]
{ \huge \bfseries A case for bioprinting organs}\\[0.4cm] % Title of your document
\HRule \\[1.5cm]

\begin{minipage}{0.4\textwidth}
\begin{flushleft} \large
\emph{Author:}\\
Mirjam \textsc{Škarica} % Your name
\end{flushleft}
\end{minipage}
~
\begin{minipage}{0.4\textwidth}
\begin{flushright} \large
\emph{Supervisor:} \\
Prof.ssa Viola \textsc{Schiaffonati} % Supervisor's Name
\end{flushright}
\end{minipage}\\[4cm]

{\large \today}\\[3cm] % Date, change the \today to a set date if you want to be precise

%\includegraphics{Logo}\\[1cm] % Include a department/university logo - this will require the graphic package

\vfill % Fill the rest of the page with whitespace

\end{titlepage}

%----------------------------------------------------------------------------------------
%	TABLE OF CONTENTS
%----------------------------------------------------------------------------------------

\tableofcontents % Include a table of contents

\newpage % Begins the essay on a new page instead of on the same page as the table of contents 


%----------------------------------------------------------------------------------------
%	INTRODUCTION
%----------------------------------------------------------------------------------------

\section{Introduction} % Major section

Three dimensional (3D) printing is an additive manufacturing process for making a physical object from a three dimensional digital model. Meaning, it builds up the solid three dimensional objects by laying down many thin successive layers of a material. It's one of the fastest growing fields and on top of it, it is the driving force behind innovations in many areas such as engineering, manufacturing and medicine. Even though 3D printing has been around since the 1980s, 3D printing of biomaterials has become possible only recently and it has catalyzed the field of bioprinting said to revolutionize everything from the pharmaceutical to the health care industry. Bioprinting creates a complex three dimensional functional and viable tissue by layering living cells onto a biologically compatible scaffolding. This way living human tissues can be generated and matured into numerous different types of human tissue and organs including skin, kidney, liver and cartilage, to name a few. And the reasons are plentiful, ranging from scientific research and drug discovery all the way to transplants and regeneration i.e. growing organs or structures to heal and promote growth in the body. Bioprinting is a much more intricate process than non-biological 3D printing because it involves dealing with live cells. Meaning, issues arise in choosing the type of cells, the means of obtaining them, preserving them, and ways of constructing tissue. Not only do these technical issues give rise to a great number of ethical questions, but also the implications of using this rapidly evolving technology also facilitate ethical dilemmas in their own right. 

This paper will try to defend the sensitive subject of organ printing by tackling various problems regarding the potential misuse, overuse, under-use and selective use of the technology in various walks of life versus it's obvious benefits. It will first address the current ethical issues like the use of embryonic stem cells and experimentation on youth. Then the discussion will move on to potential ethical issues that are expected to arise with the development of this particular technology.

\newpage

%----------------------------------------------------------------------------------------
%	MAJOR SECTION 
%----------------------------------------------------------------------------------------

\section{Current considerations} % Major section

%----------------------------------------------------------------------------------------
\subsection{The stem cell dilemma} % Sub-section

This section will address the ethical dilemma regarding the use of embryonic stem cells in the bioprinting process. 
Based on \cite{OPTN data as of January 2016} more than 121,000 men, women and children in United States alone are in need of a lifesaving organ transplant. Sadly, on average 22 people die each day waiting for a transplant. The need for efficient organ and tissue production is real and bioprinting may be a solution to this pressing issue. However, for this kind of bioprinting, embryonic stem cells are the essential part of the 3D printer's bioink.

According to a study \cite{Faulkner-Jones:2013} there is an advantage to using human embryonic stem cells in biofabrication over cells that have a more specific purpose which is that these cells have the ability to self-renew and the potential to become any other type of cell. However, their versatility is shadowed by the ethical controversy surrounding the method with which they are obtained. This is due to the fact that when the stem cells are harvested from an embryo, the extraction process actually destroys the embryo. 

The NSPE\footnote{The National Society of Professional Engineers} Code of Ethics for Engineers states as one of their fundamental canons that engineers will hold paramount the safety, health, and welfare of the public. At first, use of embryonic stem cells seems to both agree and counter this canon. 
It is in accordance with it because the research and development of this field serves the public with it's potential to save an unprecedented amount of lives. 
It seemingly contradicts the canon because it supposedly requires the taking of human life to harvest the cells and therefore, not holding public's welfare, health or safety paramount. 

However, the whole discussion about whether using embryonic stem cells is moral or not arises from a general disagreement of the status of the embryo. In other words, the disagreement about the point in time in which an embryo is considered a full human being. If we don't consider an embryo human after just 4 or 5 days following fertilization, which is the time needed for the embryo to grow to the blastocyst stage when you can harvest if for stem cells \cite{Landry:2016}, then we don't really have a moral conundrum. If, on the other hand, that is not the case, I would still like to argue that as a society we have actually already accepted the practice of utilizing embryos for stem cell research by drawing a parallel with a widely available and used reproductive technology - In Vitro Fertilization (IVF); thus rendering the controversy a bit less, well, controversial. 

In IVF, a women's eggs are extracted, inseminated in a petri dish, and after 4-5 days the now embryos are transferred into the woman's uterus. Usually more embryos are made than implanted in case the fertilization doesn't work. 
Regarding the future of those unused embryos the intended parents are left with a few choices\footnote{depending on their country's laws regarding IVF}. Usually the choices include freezing, discarding or donating the surplus embryos either to other couples trying to conceive or for scientific research. 
Indeed, current sources of hESC\footnote{human embryo stem cells} for stem cell research include excess embryos that would have been discarded as an unavoidable part of the IVF process\cite{Pediatrics:2012}.

Regarding the faith of the frozen embryos, the ones that never get used are usually eventually thawed and discarded. On the other hand, according to one study's \cite{Pavone:2011} compiled statistics on embryos that do get thawed for reuse, 86\% of those are discarded either because they didn't survive the thawing process or just because they didn't get implanted this time around. 

Furthermore, it is important to realize that in developing the IVF procedure a great number of embryos must have been used and dispensed, essentially sacrificed for science. 

Having all this in mind it is relevant to acknowledge the fact society widely accepts and uses IVF. One might go far as to say that it is considered one of the modern medical triumphs. Hence it can be concluded that by such doing, and the numbers back this claim up, regardless of whether or not we consider embryos human, we have deemed all the past and continuous "sacrifice" of embryos worthy of a greater cause. In the case of IVF the cause is creating a life.

Just like helping people conceive a child is a noble cause, so too are the efforts made not only to save people's lives by bioprinting an organ they desperately need, but also by reducing the need for human donors. And to take things to an extreme, if we could remove all together the need for adult human donors, we would end other moral and ethical dilemmas surrounding organ transplants today. The dilemma about requiring a life taken for a live given, and the issue of having to choose selectively to save one life over another. 

The development of the bioprinting field presents us with tremendous opportunities to better and to save lives. So, in my opinion, it is more wrong to stifle research with such promising medical potential over a dilemma regarding a procedure that was and in fact currently is in practice.



%----------------------------------------------------------------------------------------
\subsection{Human trials} % Sub-section

Bioprinting is a natural, albeit mind-altering, development within the 3D printing field. And the latter one has already yielded many successful surgical interventions. Just to name a few, in 2015, three infants with a life threatening airway disease were saved by 3D printing a flexible tracheal\footnotemark{} splints tailored for each individual patient \cite{Morrison:2015}.
\footnotetext{The trachea, colloquially windpipe, is a tube that connects the larynx to the bronchi of the lungs. It is an integral part of the body’s airway and has the vital function of providing air flow to and from the lungs for respiration}

Another miracle story is that of a three year old girl born with hydrocephalus\footnotemark{} who was admitted to the hospital after her brain suffered serious infections and had a high probability of rupture due to her swollen skull. She underwent an unheard of life saving surgery consisting of constructing and inserting a new 3D printed skull \cite{3dprint.com:2015}. 
\footnotetext{Hydrocephalus is a medical condition that occurs when too much cerebrospinal fluid builds up in the skull causing it to swell}

The success of these stories is enough to get one excited about the future and want to go the extra mile and start the bioprinting trials. However printing and implanting biodegradable cartilages is a far simpler endeavor than bioprinting personalized human tissue. 

One of the bigger challenges being that testing for safety is limited. For one, it is impossible to compile a reliable statistic regarding a newly printed tissue. This is due to the fact that testing the same tissue on a large number of healthy people cannot, by definition, be done because the tissue is personalized and made specially for each individual.

This raises questions about the ethicalness of offering treatment to desperate patients not fully understanding the consequences and without knowing the outcome. While this can be morally justified by offering the procedure only to people in dire need and through the process of informed consent, there might be a greater danger looming behind closed doors. 

According to \cite{Army Technology:2014} publication, The US Army is heavily investing and is hoping to soon begin clinical trials with 3D printed skin. Given the military nature of the organization, it is safe to assume the trials will be held in secrecy, better known as, confidentially. While this bit of information was made available to the public, we could infer that it is just the tip of the proverbial iceberg and that many more trials are performed by many more government bodies all over the word. If history has shown us anything it is that government agencies don't shy away from using less than ethical methods in the race to the top. Keeping the public in the dark about innovations, even if the reasons are their own safety, still leaves the people in a very vulnerable position.

Unfortunately, things have always been this way, armies have always had secret projects and a lot can be said about how actually informed is informed consent. Not to say challenging the status quo shouldn't be done, but it is to say that this is an all pervasive problem, not strictly tied to bioprinting. What is even more discouraging is the staggeringly low rate of which animal trials translate to human trials \cite{Worp:2010}. In other words, the animal models do not predict with sufficient certainty what will happen in a human body once a new drug is tried. In fact, at the time of writing, BBC News had reported an unfortunate outcome of a clinical trial in France. A new drug, a cannabis based painkiller, has left one person brain-dead and another five people in hospital, in a serious condition \cite{BBC News:2016}. Meaning that in spite of our best moral and technical efforts directed to minimize the risk of clinical trials, we are failing to protect people entering them.

However, there may be light at the end of the tunnel and that light is bioprinting. It is stand to usher in a new era in medicine and pharmaceutical industry. An era where drug testing therapies will be done on living human tissue instead of animals. Indeed, this is very close to becoming a commercial reality. A pioneering biotech company Organovo has already made great strides in doing so. In 2014 they have announced commercial release of the exVive3D human liver tissue for preclinical drug discovery testing \cite{Organovo:2014}, or popularly called by the press, a mini-liver. These models offer a more accurate and reproducible tool for answering questions related to human biology at the tissue level \cite{Visk:2015}. By printing miniature livers, pharmaceutical companies will be able to obtain a human-specific data and make a better, more informed prediction of how drugs might affect a human patient before going into risky clinical trials. They had another notable accomplishment in 2015, when they announced a research partnership with  L'Oreal USA aimed in developing 3D bioprinted skin tissue \cite{Organovo:2015}. A partnership which they hope will not only end the need for the controversial, if not cruel, animal testing; but will also bring about new advanced, affordable and reliable in vitro methods for evaluating product safety and performance.

The potential to build human tissue models to study the development and progression of disease provides an enormous opportunity to enhance drug discovery and development by reducing cost, time and risk associated with the later stages of clinical trials \cite{Visk:2015}. As the technology advances we could see whole human body systems being modeled by connecting printed organs and tissues, everything from digestive and lymphatic system all the way to the endocrine and muscular system. This could lead to an even more comprehensive and reliable picture of the new treatment's benefits by observing how experimental drugs are metabolized and, more importantly, what side effects they might have on the human body. Moreover, down the line we can expect individualized treatments and drugs delivered to patients. Where treatment is first tested and drugs finely tuned on a personalized model of the patient's own body. 

With all this said, it is easy to imagine that bioprinting will improve our quality of life by revolutionizing the health-care industry for the better, repairing our bodies, and even reducing or eliminating the need for using of animals in pharmaceutical and medical testing.


\newpage 

%----------------------------------------------------------------------------------------
%	MAJOR SECTION 
%----------------------------------------------------------------------------------------

\section{Future considerations} % Major section

Organ bioprinting may still be a ways off in the future, but is certainly not impossible. The further development of the technology will raise a number of ethical questions that need to be considered carefully now. Now, because there is still time to discuss the inevitable future in order to be able to usher in a new era prepared, instead of getting caught by surprise and needing to deal with the aftermath of the biotechnological boom. 

%------------------------------------------------

\subsection{Access and regulatory debate} % Sub-section

Bioprinting field is advancing at breakneck speed which is not something you could generally say for legislation. The fear is that the research going on worldwide will outpace regulatory agencies and the lack of general understanding and consequences of this technology will leave the public unprotected and vulnerable for misuse. 

Currently, any interested research group is building their own 3D bioprinter. Having a tinker prone mindset is completely fine, if not encouraging, due to the research nature of the emerging field. But as the field advances the need for regularization and standardization will be of paramount importance. This will not be an easy task as there are a lot of moving parts that make up organ bioprinting. We need to think about how to license and regulate the production and distribution of organ printers and once we've done that, how do we ensure they are not tampered with or added upon? 
Another problem is the distribution and quality control of the bioink being used so the patients don't end up with sub-par organs. If we leave these issues to self-regulation or, similarly, no-regulation, we might end up with a lot of DIY\footnote{Do It Yourself} disasters. Clearly self or no regulation is not the way to go.

Moreover, how will we handle the matter of intellectual property, copyrights of all: organ printers, bioprinting scaffolds, bioink materials along with their ratios, and the design of organs?  This is to say that we need to deliberate on the issue\footnotemark{} of patentability of printed organs. 
\footnotetext{America Invents Act states that "no patent may issue on a claim directed to or encompassing a human organism"}
Because we will soon have a high-grossing industry on our hands, a business. And this is the kind of business we don't want to leave to free interpretations of corporations. 
Once this is sorted, how do we implement and regulate the waiting lists for new organ transplants and who will cover the costs? Will the government health plans include emergency organ printing or will it fall upon the people to deal with the costs themselves? 

Ideally, it shouldn't be possible for us to just buy new organs, because then the wage gap will truly drill open the chasm of privilege of unprecedented depth. Unfortunately, we live in a world that's far from ideal. There already are many privately owned clinics that are successful businesses in their own right which in itself isn't a bad thing. 
On the contrary, it could be argued that those clinics in fact lighten the load of the public heath-care systems all over the world. But once bioprinting organs becomes a reality, a system where the wealthy can simply wish and pay for the organ they need, while the less well-off are stuck in a seemingly endless line, patiently waiting for their turn, might become a morally questionable practice. We will need to think of a way to regulate the circumstance under which a person is legally allowed to have a new organ printed to avoid creating a death-cheating long-lived high-class and a short-lived lower class of people.

Fortunately, there exist propositions concerning regulation of bioprinting that, with careful and timely planning, might mitigate the unwanted repercussions of these concerns. One such proposition suggests regulating the commercial use of bioprinting (i.e. not including search) should be done by both, government agencies and medical regulatory bodies. The government agencies should control access to materials that make up bioink and to the organ designs. Also they should regulate who can manufacture organ bioprinters as well as who can obtain them. Only health care institutions who have medical professionals trained this particular the field should be able to acquire bioprinters as long as their work is monitored closely by medical regulatory bodies. Hopefully this approach will prevent misuse, whether voluntary or involuntary, while still allowing the development and research to continue.

%------------------------------------------------

\subsection{Human enhancement} % Sub-section

So far we have enjoyed and been a subject of natural selection, an evolution of the Darwinian kind. Now it is hypothesized that we are on the cusp of seeing a new kind of evolution emerging, the evolution by design. One of such instances of evolution by design is intentional human enhancement. So far, only the efforts of printing functional replicas of organs which mimic human biology were being discussed. And while this is a worthwhile topic, simply replacing failed organs is thinking small. We could be able to engineer better, more useful organs and tissue. Sure, the human hart is an amazing pump but is it the best possible one?  After all, so many people \cite{Wood:2002} are already walking around with a small electric device inside their chests to help control abnormal heart rhythm - a pacemaker; because their own heart has failed them.

Clearly there is room and a need for improvement. And it seems that researchers at the Advanced Manufacturing Technology (AMTech) group at the University of Iowa couldn't agree more. They have been diligently developing a glucose-sensitive pancreatic organ that can be 3D printed and grown in a laboratory \cite{Ozbolat:2013}. Which can then be transplanted anywhere inside the body to help regulate the glucose levels in blood of diabetic patients. This new organ won't be an exact replica of an ordinary human pancreas, but a modified, or better yet an enhanced version of it. The researchers' goal is to cure diabetes by having the organ printed with the patient's own cells instead of just treating or keeping in check patient's diabetes with drugs.

The AMTech group's crusade for ending diabetes is evidently led by good intentions, but as the old saying goes, the road to hell is paved with good intentions. Bioprinted engineered organs which don't exist in nature open an entire new realm of possibilities, and with that, problems. 
Human enhancement can range from increasing quality of life, extending the lifespan of a person by continually reprinting and replacing dying or damaged organs, all the way to creating super humans. There is merit in wondering how will our behavior and habits aimed toward keeping our health change as a result of being able to print a new organ on demand. Will smoking be as frowned upon when lungs can be replaced with a healthy replica or an enhanced one with better filtering ability? What about our stance regarding excessive drinking and diets high in sugar, considering new liver or an extra pancreatic like organ are just a routine surgery away. Will this kind of progress change the face of sports where one's athletic ability is only limited by technology and by her access to the state of the art muscle tissue, enhanced bones or more adaptable lungs.

Allow me to push the discussion to it's limits and entertain the possibility that one day we will able to print an entire human. It is difficult to even imagine all the repercussions of such capability. How will our value for human life change? Most of our laws and shared morality stem from protecting human life and regarding it as priceless. Will we still be able to justify and cling to the idea of human life being priceless even though we'll have the intelligence to precisely calculate the cost of printing a human being? What is even scarier is the potential military applications of having the capability to design and print human bodies. The sheer idea of a disposable superhuman army is enough to make anyone shiver.
 
Thankfully and hopefully, the latter cogitations are so far off that they add no real pragmatic value to current discussions. 
The minor and beneficial interventions, however, are likely to become available as soon as the organ bioprinting technology matures enough. And if the current popularity of cosmetic surgery is anything to go by, humans will flock to enhancing and replacing their body parts for sometimes morally and utilitarianly questionable reasons. Since we are already allowing people to alter their bodies in a way nature did not intend for far more superfluous reasons than saving a life; I see no reason why we shouldn't do the same with surgical interventions of far greater value. And to be fair, aesthetics are not the only reason people get things implanted. As mentioned before, we are already augmenting the human body by surgically inserting pacemakers, implantable defibrillators and neurostimulators. Except future techniques would be much safer, less invasive and arguably more natural than their current crude mechanical counterparts.

\newpage 
%----------------------------------------------------------------------------------------
%	CONCLUSION
%----------------------------------------------------------------------------------------

\section{Conclusion} % Major section

Bioprinted organs are set to change our lives for the better by improving the efficiency, safety and cost of drug trials, by personalizing medical treatment, by cutting down waiting for organ transplants, by reducing risk of rejection of transplanted organs and even by eradicating diseases.  

Bioprinting truly is a modern wonder but, at the risk of sounding cliché, with great power comes great responsibility. 
If history has tough us anything, it's that stifling innovation never works in the long run. Therefore the best thing we can do is to think long and hard about the problems, or better yet - the opportunities, in order to prepare ourselves by raising awareness, coming up with best practices, tweaking our laws and putting in place an infrastructure to better accommodate this imminent change.

\newpage

%----------------------------------------------------------------------------------------
%	BIBLIOGRAPHY
%----------------------------------------------------------------------------------------

\begin{thebibliography}{99} % Bibliography - this is intentionally simple in this template
\begin{small}

%------------------------------------------------------------------
\bibitem[3dprint.com (2015)]{3dprint.com:2015}
\newblock http://3dprint.com/81815/3d-printed-skull/

%------------------------------------------------------------------
\bibitem[Army Technology (2014)]{Army Technology:2014}
\newblock Dan Lafontaine, Rdecom public affairs (2014).
\newline Medical Applications for 3-D. Army invests in 3-D bioprinting to treat injured Soldiers
\newblock {\em Army Technology publication}
\newblock http://www.army.mil/e2/c/downloads/353394.pdf

%------------------------------------------------------------------
\bibitem[BBC News (2016)]{BBC News:2016}
\newblock http://www.bbc.com/news/world-europe-35320895

%------------------------------------------------------------------
\bibitem[Faulkner-Jones et al. (2013)]{Faulkner-Jones:2013}
\newblock Alan Faulkner-Jones and Sebastian Greenhough and Jason A King and John Gardner and Aidan Courtney and Wenmiao Shu (2013).
\newline Development of a valve-based cell printer for the formation of human embryonic stem cell spheroid aggregates
\newblock {\em Biofabrication}

%------------------------------------------------------------------
\bibitem[Landry et al. (2016)]{Landry:2016}
\newblock Landry, Donald W., and Howard A. Zucker. (2016).
\newline Embryonic Death and the Creation of Human Embryonic Stem Cells
\newblock {\em Journal of Clinical Investigation}

%------------------------------------------------------------------
\bibitem[Morrison et al. (2015)]{Morrison:2015}
\newblock Morrison, Robert J. and Hollister, Scott J. and Niedner, Matthew F. and Mahani, Maryam Ghadimi and Park, Albert H. and Mehta, Deepak K. and Ohye, Richard G. and Green, Glenn E. (2015).
\newline Mitigation of tracheobronchomalacia with 3D-printed personalized medical devices in pediatric patients
\newblock {\em Science Translational Medicine}

%------------------------------------------------------------------
\bibitem[Murphy and Atala (2014)]{MurphyAtala:2014}
\newblock S. V. Murphy and A. Atala (2014).
\newline {3D bioprinting of tissues and organs}
\newblock {\em Nat Biotech} 

%------------------------------------------------------------------
\bibitem[OPTN data as of January 2016]{OPTN data as of January 2016}
\newblock The Organ Procurement and Transplantation Network.
\newblock http://optn.transplant.hrsa.gov

%------------------------------------------------------------------
\bibitem[Organovo press release (2014)]{Organovo:2014}
\newblock http://ir.organovo.com/news/press-releases/press-releases-details/2014/Organovo-Announces-Commercial-Release-of-the-exVive3D-Human-Liver-Tissue/

%------------------------------------------------------------------
\bibitem[Organovo press release (2015)]{Organovo:2015}
\newblock http://ir.organovo.com/news/press-releases/press-releases-details/2015/LOreal-USA-Announces-Research-Partnership-with-Organovo-to-Develop-3-D-Bioprinted-Skin-Tissue/

%------------------------------------------------------------------
\bibitem[Ozbolat et al. (2013)]{Ozbolat:2013}
\newblock Ibrahim T. Ozbolat and Yin Yu (2013).
\newline Bioprinting Toward Organ Fabrication: Challenges and Future Trends
\newblock {\em IEEE transactions on biomedical engineering}

%------------------------------------------------------------------
\bibitem[Pavone et al. (2011)]{Pavone:2011}
\newblock Pavone, M. E., Innes, J., Hirshfeld-Cytron, J., Kazer, R., and Zhang, J. (2011).
\newline Comparing thaw survival, implantation and live birth rates from cryopreserved zygotes, embryos and blastocysts. 
\newblock {\em Journal of Human Reproductive Sciences}
\newblock http://doi.org/10.4103/0974-1208.82356

%------------------------------------------------------------------
\bibitem[Pediatrics (2012)]{Pediatrics:2012}
\newblock Committee for pediatric research and committee on bioethics  (2012).
\newline Human Embryonic Stem Cell (hESC) and Human Embryo Research
\newblock {\em Pediatrics}
\newblock http://pediatrics.aappublications.org/content/130/5/972

%------------------------------------------------------------------
\bibitem[Visk (2015)]{Visk:2015}
\newblock DeeAnn Visk (2015).
\newline Will Advances in Preclinical In VitroModels Lower the Costs of Drug Development?
\newblock {\em Applied In Vitro Toxicology}

%------------------------------------------------------------------
\bibitem[Wood et al. (2002)]{Wood:2002}
\newblock Mark A. Wood, MD; Kenneth A. Ellenbogen, MD(2002).
\newline Cardiac Pacemakers From the Patient’s Perspective
\newblock {\em Circulation}

%------------------------------------------------------------------
\bibitem[Worp et al. (2010)]{Worp:2010}
\newblock van der Worp, H. Bart AND Howells, David W. AND Sena, Emily S. AND Porritt, Michelle J. AND Rewell, Sarah AND O'Collins, Victoria AND Macleod, Malcolm R
(2010). 
\newline Can Animal Models of Disease Reliably Inform Human Studies?
\newblock {\em PLoS Med}
\newblock http://journals.plos.org/plosmedicine/article?id=10.1371/journal.pmed.1000245


\end{small}
\end{thebibliography}

%----------------------------------------------------------------------------------------

\end{document}
