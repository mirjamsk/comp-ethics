%%%%%%%%%%%%%%%%%%%%%%%%%%%%%%%%%%%%%%%%%
% Simple Sectioned Essay Template
% LaTeX Template
%
% This template has been downloaded from:
% http://www.latextemplates.com
%
%
%%%%%%%%%%%%%%%%%%%%%%%%%%%%%%%%%%%%%%%%%

%----------------------------------------------------------------------------------------
%	PACKAGES AND OTHER DOCUMENT CONFIGURATIONS
%----------------------------------------------------------------------------------------

\documentclass[12pt]{article} % Default font size is 12pt, it can be changed here
\usepackage[utf8]{inputenc}
\usepackage{geometry} % Required to change the page size to A4
\geometry{a4paper} % Set the page size to be A4 as opposed to the default US Letter

\usepackage{graphicx} % Required for including pictures

\usepackage{float} % Allows putting an [H] in \begin{figure} to specify the exact location of the figure
\usepackage{wrapfig} % Allows in-line images such as the example fish picture

\usepackage{lipsum} % Used for inserting dummy 'Lorem ipsum' text into the template

\linespread{1.2} % Line spacing

%\setlength\parindent{0pt} % Uncomment to remove all indentation from paragraphs

\graphicspath{{Pictures/}} % Specifies the directory where pictures are stored

\begin{document}

%----------------------------------------------------------------------------------------
%	TITLE PAGE
%----------------------------------------------------------------------------------------

\begin{titlepage}

\newcommand{\HRule}{\rule{\linewidth}{0.5mm}} % Defines a new command for the horizontal lines, change thickness here

\center % Center everything on the page

\textsc{\LARGE Politecnico di Milano}\\[1.5cm] % Name of your university/college
\textsc{\large Computer ethics}\\[0.5cm] % Minor heading such as course title

\HRule \\[0.4cm]
{ \huge \bfseries A case for bioprinting organs}\\[0.4cm] % Title of your document
\HRule \\[1.5cm]

\begin{minipage}{0.4\textwidth}
\begin{flushleft} \large
\emph{Author:}\\
Mirjam \textsc{Škarica} % Your name
\end{flushleft}
\end{minipage}
~
\begin{minipage}{0.4\textwidth}
\begin{flushright} \large
\emph{Supervisor:} \\
Prof.ssa Viola  \textsc{Schiaffonati} % Supervisor's Name
\end{flushright}
\end{minipage}\\[4cm]

{\large \today}\\[3cm] % Date, change the \today to a set date if you want to be precise

%\includegraphics{Logo}\\[1cm] % Include a department/university logo - this will require the graphic package

\vfill % Fill the rest of the page with whitespace

\end{titlepage}

%----------------------------------------------------------------------------------------
%	TABLE OF CONTENTS
%----------------------------------------------------------------------------------------

\tableofcontents % Include a table of contents

\newpage % Begins the essay on a new page instead of on the same page as the table of contents 


%----------------------------------------------------------------------------------------
%	INTRODUCTION
%----------------------------------------------------------------------------------------

\section{Introduction} % Major section

Three dimensional (3D) printing is an additive manufacturing process for making a physical object from a three dimensional digital model. Meaning, it builds up the solid three dimensional objects by laying down many thin successive layers of a material. It's one of the fastest growing fields and on top of it, it is the driving force behind innovations in many areas such as engineering, manufacturing and medicine. Even though 3D printing has been around since the 1980s, 3D printing of biomaterials has become possible only recently and it has catalyzed the field of bioprinting said to revolutionize everything from the pharmaceutical to the health care industry. Bioprinting creates a complex three dimensional functional and viable tissue by layering living cells onto a biologically compatible scaffolding. Numerous tissues can be generated this way. And the reasons are plentiful, ranging from scientific research and drug discovery all the way to transplants and regeneration i.e. growing organs or structures to heal and promote growth in the body. Bioprinting is a much more intricate process than non-biological 3D printing because it involves dealing with live cells. Meaning, issues arise in choosing the type of cells, the means of obtaining them, preserving them, and ways of constructing tissue. Not only do these technical issues give rise to a great number of ethical questions, but also the implications of using this rapidly evolving technology also facilitate ethical dilemmas in their own right. 

This paper will try to defend the sensitive subject of organ printing by tackling various problems regarding the potential misuse, overuse, under-use and selective use of the technology in various walks of life versus it's obvious benefits. It will first address the current ethical issues like the use of embryonic stem cells and experimentation on youth. Then the discussion will move on to potential ethical issues that are expected to arise with the development of this particular technology.

\newpage

%----------------------------------------------------------------------------------------
%	MAJOR SECTION 
%----------------------------------------------------------------------------------------

\section{Current issues} % Major section

%----------------------------------------------------------------------------------------
\subsection{The stem cell dilemma} % Sub-section

This section will address the ethical dilemma regarding the use of embryonic stem cells in the bioprinting process. 
Based on \cite{OPTN data as of January 2015} more than 123,000 men, women and children in United States alone are in need of a lifesaving organ transplant. More than 1 million tissue transplants are done each year. But sadly, on average 22 people die each day waiting for a transplant. The need for efficient organ and tissue production is real. Bioprinting may be a solution to this pressing issue by using stem cells as the essential part of 3D printer's bioink.

According to a study \cite{Faulkner-Jones:2013} there is an advantage to using human embryonic stem cells in biofabrication over cells that have a more specific purpose which is that these cells have the ability to self-renew and the potential to become any other type of cell. However, their versatility is shadowed by the ethical controversy surrounding the method with which they are obtained. This is mainly because the stem cells are harvested from an embryo and the extraction process actually destroys the embryo. 

The NSPE\footnote{The National Society of Professional Engineers} Code of Ethics for Engineers states as one of their fundamental canons that engineers will hold paramount the safety, health, and welfare of the public. At first, use of embryonic stem cells seems to both agree and counter this canon. 
It agrees because the research and development of this field serves the public with it's potential to save unprecedented amount of lives. 
It seemingly contradicts the canon because it supposedly requires the taking of human life to harvest the cells and therefore, not holding public's welfare, health or safety paramount. 


However, the whole discussion about whether or not using embryonic stem cells is moral arises from a general disagreement of the status of the embryo. In other words, what is the point an embryo is considered a full human. 

I'd agrue harvesting embryos is not an unethical process. In my opinion nije još human, a i extendala bi to toliko da se u principu i javnost sa mnom slaže zbog opce prihvacenog IVFa


But even without agreeing on the status of embryo, and condemning the IVF methods as barbaric, the discussion could boil down to is 
how ethical it is to use a live for a life. But that is the same dilemma we have with the transplants, a practice we have been using since the 19xxs  and deemed worthy the risk. 

The potential black market (illigal, shady ways of obtaining stem cells) has the same 'blahblah' as the black market in organ transplates
---


TIt is in line with the canon because, with health and welfare as the main focus, the use of these cells fosters the development of the medical field, which indicates that the public is in the best interest.  It also opposes the canon because it requires the taking of a human life to obtain the cells to print th




Extracting stem cells from embryos is not an unethical process. The embryos used are just four or five days old– essentially created by squirting semen onto a petri dish with some human egg cells on it, as in IVF treatment, a scientific technique which is considerably less contentious. Indeed, ES cells are often taken from excess IVF embryos that would otherwise have been discarded. The extraction process destroys the embryo, but at this stage it consists of a rough mass of less than a hundred cells and is unrecognisable as human, and more akin to a cauliflower. 

not more unethical than IVF?? if IFV discards other embryo

Some believe that terminating embryos is inherently immoral because life begins at fertilisation, making such an act is equivalent to murder. I sometimes wonder if these people know that, even in healthy human females, three-quarters of all fertilised eggs are spontaneously miscarried and fail to survive to birth. as many as half do not even implant into the mother’s uterus. If your ethical system counts the destruction of any fertilised egg as murder, the majority of women on the planet would have to be put behind bars.

There is nothing unethical about using embryonic stem cells. It is far more wrong, in my opinion, to block research with such demonstrable medical potential.

The use of embryonic stem cell, while controversial, would in fact advance the health and welfare of the public because they would quicken the time it will take to successfully transplant biofabricated organs and reduce the need for human donors.



In developing and continuously improving IVF we must suppose that thousands of embryos were used and discarted
 and for the most part IVF is accepted and agrred as being a good thing


Ovisno o poziciji covjeka, ako se misli da je life == in the moment of conception onda se postavlja se pitanje vredujemo li jedan zivot vise od drugoga. ali 
to pitanje se vec postavlja trenutnom transpant sustavom gjdje pacijent ceka drugog da umre. 
ugl treba paziti da se ne abortira u službu harvestanja stem stanica onda je ok.
takodjer se treba paziti da su info disclosed
pharmaceutical testing 

%----------------------------------------------------------------------------------------
\subsection{Youth experimentation} % Sub-section
ima dijete na kojem je isprinatn onaj dišni cartlige


\newpage 

%----------------------------------------------------------------------------------------
%	MAJOR SECTION 
%----------------------------------------------------------------------------------------

\section{Future issues} % Major section

\subsection{Human inhancement} % Sub-section
When organ printing really is in full swing
http://3dprint.com/28879/3d-printing-new-unique-organs/

superhumans
athletes

%------------------------------------------------

\subsection{Safety} % Sub-section
open source, or just source

%------------------------------------------------

\subsection{Justice and access} % Sub-section
selective organs transplants,
wage gap,


\newpage 

%----------------------------------------------------------------------------------------
%	CONCLUSION
%----------------------------------------------------------------------------------------

\section{Conclusion} % Major section

Bioprinting truly is a modern wonder but, at the risk of sounding cliché, with great power comes great responsibility. If history has tough us anything, it's that stifling innovation never works in the long run. Therefore the best thing we can do is to think long and hard about the problem, or better yet - the opportunity, in order to prepare ourselves by raising awareness, coming up with best practices, tweaking our laws, putting in place an infrastructure to better accommodate this imminent change.

\newpage

%----------------------------------------------------------------------------------------
%	BIBLIOGRAPHY
%----------------------------------------------------------------------------------------

\begin{thebibliography}{99} % Bibliography - this is intentionally simple in this template

%------------------------------------------------------------------
\bibitem[Murphy and Atala (2014)]{MurphyAtala:2014}
\newblock S. V. Murphy and A. Atala (2014).
\newblock {3D bioprinting of tissues and organs}
\newblock {\em Nat Biotech} 

%------------------------------------------------------------------
\bibitem[OPTN data as of January 2015]{OPTN data as of January 2015}
\newblock The Organ Procurement and Transplantation Network.
\newblock http://optn.transplant.hrsa.gov

%------------------------------------------------------------------
\bibitem[Faulkner-Jones et al. (2013)]{Faulkner-Jones:2013}
\newblock Alan Faulkner-Jones and Sebastian Greenhough and Jason A King and John Gardner and Aidan Courtney and Wenmiao Shu (2013).
\newblock Development of a valve-based cell printer for the formation of human embryonic stem cell spheroid aggregates
\newblock {\em Biofabrication}

%------------------------------------------------------------------
\bibitem[Tony Dyson (1998)]{Dyson:1998}
\newblock Tony Dyson (1998).
\newblock Ethics Of In Vitro Fertilisation (Ethics, Our Choices)

%------------------------------------------------------------------

\end{thebibliography}

%----------------------------------------------------------------------------------------

\end{document}
