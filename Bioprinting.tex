%%%%%%%%%%%%%%%%%%%%%%%%%%%%%%%%%%%%%%%%%
% Simple Sectioned Essay Template
% LaTeX Template
%
% This template has been downloaded from:
% http://www.latextemplates.com
%
%
%%%%%%%%%%%%%%%%%%%%%%%%%%%%%%%%%%%%%%%%%

%----------------------------------------------------------------------------------------
%	PACKAGES AND OTHER DOCUMENT CONFIGURATIONS
%----------------------------------------------------------------------------------------

\documentclass[12pt]{article} % Default font size is 12pt, it can be changed here

\usepackage{geometry} % Required to change the page size to A4
\geometry{a4paper} % Set the page size to be A4 as opposed to the default US Letter

\usepackage{graphicx} % Required for including pictures

\usepackage{float} % Allows putting an [H] in \begin{figure} to specify the exact location of the figure
\usepackage{wrapfig} % Allows in-line images such as the example fish picture

\usepackage{lipsum} % Used for inserting dummy 'Lorem ipsum' text into the template

\linespread{1.2} % Line spacing

%\setlength\parindent{0pt} % Uncomment to remove all indentation from paragraphs

\graphicspath{{Pictures/}} % Specifies the directory where pictures are stored

\begin{document}

%----------------------------------------------------------------------------------------
%	TITLE PAGE
%----------------------------------------------------------------------------------------

\begin{titlepage}

\newcommand{\HRule}{\rule{\linewidth}{0.5mm}} % Defines a new command for the horizontal lines, change thickness here

\center % Center everything on the page

\textsc{\LARGE Politecnico di Milano}\\[1.5cm] % Name of your university/college
\textsc{\large Computer ethics}\\[0.5cm] % Minor heading such as course title

\HRule \\[0.4cm]
{ \huge \bfseries A case for bioprinting organs}\\[0.4cm] % Title of your document
\HRule \\[1.5cm]

\begin{minipage}{0.4\textwidth}
\begin{flushleft} \large
\emph{Author:}\\
Mirjam \textsc{Skarica} % Your name
\end{flushleft}
\end{minipage}
~
\begin{minipage}{0.4\textwidth}
\begin{flushright} \large
\emph{Supervisor:} \\
Prof.ssa Viola  \textsc{Schiaffonati} % Supervisor's Name
\end{flushright}
\end{minipage}\\[4cm]

{\large \today}\\[3cm] % Date, change the \today to a set date if you want to be precise

%\includegraphics{Logo}\\[1cm] % Include a department/university logo - this will require the graphicx package

\vfill % Fill the rest of the page with whitespace

\end{titlepage}

%----------------------------------------------------------------------------------------
%	TABLE OF CONTENTS
%----------------------------------------------------------------------------------------

\tableofcontents % Include a table of contents

\newpage % Begins the essay on a new page instead of on the same page as the table of contents 

%----------------------------------------------------------------------------------------
%	INTRODUCTION
%----------------------------------------------------------------------------------------

\section{Introduction} % Major section


Three dimensional (3D) printing is an additive manufacturing process for making a physical object from a three dimensional digital model. Meaning, it builds up the solid three dimensional objects by laying down many thin successive layers of a material. It's one of the fastest growing fields and on top of it, it is the driving force behind innovations in many areas such as engineering, manufacturing and medicine. Even though 3D printing has been around since the 1980s, 3D printing of biomaterials has become possible only recently and it has catalyzed a field said to revolutionize the everything from the pharmaceutical to the healthcare industry - bioprinting. Bioprinting creates a complex three dimensional functional and viable tissue by layering living cells onto a biologically compatible scaffolding. Numerous tissues can be generated this way. And the reasons are plentiful, ranging from scientific research and drug discovery all the way to transplants and regeneration i.e. growing organs or structures to heal and promote growth in the body. Bioprinting is much more intricate process than non-biological 3D printing because it involves dealing with live cells. Meaning, issues arise in choosing the type of cells, the means of obtaining them, preserving them, and constructing tissue. Not only do these technical issues give rise to great number of ethical questions, but also the implications of using this rapidly evolving technology also facilitate ethical dilemmas in their own right. 

Bioprinting truly is a modern wonder but, at the risk of sounding cliché, with great power comes great responsibility. This paper will try to defend the sensitive subject of organ printing by tackling various problems regarding the potential misuse, overuse, under-use and selective use of the technology in various walks of life versus it's obvious benefits. Because if history has tough us anything, it's that stifling innovation never works in the long run. Therefore the best thing we can do is to think long and hard about the problem, or better yet, the opportunity in order to prepare ourselves by coming up with best practices, tweaking our laws, putting in place an infrastructure to better accommodate this imminent change.

%------------------------------------------------

\subsection{Subsection 1} % Sub-section

\lipsum[1] % Dummy text

%------------------------------------------------

\subsection{Subsection 2} % Sub-section

\lipsum[2] % Dummy text

%------------------------------------------------

\subsubsection{Subsubsection 1} % Sub-sub-section

\lipsum[3] % Dummy text

\begin{figure}[H] % Example image
\center{\includegraphics[width=0.5\linewidth]{placeholder}}
\caption{Example image.}
\label{fig:speciation}
\end{figure}

%------------------------------------------------

\subsubsection{Subsubsection 2} % Sub-sub-section

\lipsum[4] % Dummy text

%----------------------------------------------------------------------------------------
%	MAJOR SECTION 1
%----------------------------------------------------------------------------------------

\section{Content Section} % Major section

\lipsum[5] % Dummy text

%------------------------------------------------

\subsection{Subsection 1} % Sub-section

\subsubsection{Subsubsection 1} % Sub-sub-section

\lipsum[6] % Dummy text

%------------------------------------------------

\subsubsection{Subsubsection 2} % Sub-sub-section

\lipsum[6] % Dummy text
\begin{wrapfigure}{l}{0.4\textwidth} % Inline image example
  \begin{center}
    \includegraphics[width=0.38\textwidth]{fish}
  \end{center}
  \caption{Fish}
\end{wrapfigure}
\lipsum[7-8] % Dummy text

%------------------------------------------------

\subsubsection{Subsubsection 3} % Sub-sub-section

\begin{description} % Numbered list example

\item[First] \hfill \\
\lipsum[9] % Dummy text

\item[Second] \hfill \\
\lipsum[10] % Dummy text

\item[Third] \hfill \\
\lipsum[11] % Dummy text

\end{description} 

%----------------------------------------------------------------------------------------
%	MAJOR SECTION X - TEMPLATE - UNCOMMENT AND FILL IN
%----------------------------------------------------------------------------------------

%\section{Content Section}

%\subsection{Subsection 1} % Sub-section

% Content

%------------------------------------------------

%\subsection{Subsection 2} % Sub-section

% Content

%----------------------------------------------------------------------------------------
%	CONCLUSION
%----------------------------------------------------------------------------------------

\section{Conclusion} % Major section

\lipsum[12-13]

%----------------------------------------------------------------------------------------
%	BIBLIOGRAPHY
%----------------------------------------------------------------------------------------

\begin{thebibliography}{99} % Bibliography - this is intentionally simple in this template

\bibitem[Figueredo and Wolf, 2009]{Figueredo:2009dg}
Figueredo, A.~J. and Wolf, P. S.~A. (2009).
\newblock Assortative pairing and life history strategy - a cross-cultural
  study.
\newblock {\em Human Nature}, 20:317--330.
 
\end{thebibliography}

%----------------------------------------------------------------------------------------

\end{document}
