%%%%%%%%%%%%%%%%%%%%%%%%%%%%%%%%%%%%%%%%%
% Simple Sectioned Essay Template
% LaTeX Template
%
% This template has been downloaded from:
% http://www.latextemplates.com
%
%
%%%%%%%%%%%%%%%%%%%%%%%%%%%%%%%%%%%%%%%%%

%----------------------------------------------------------------------------------------
%	PACKAGES AND OTHER DOCUMENT CONFIGURATIONS
%----------------------------------------------------------------------------------------

\documentclass[12pt]{article} % Default font size is 12pt, it can be changed here
\usepackage[utf8]{inputenc}
\usepackage{geometry} % Required to change the page size to A4
\geometry{a4paper} % Set the page size to be A4 as opposed to the default US Letter

\usepackage{graphicx} % Required for including pictures

\usepackage{float} % Allows putting an [H] in \begin{figure} to specify the exact location of the figure
\usepackage{wrapfig} % Allows in-line images such as the example fish picture

\usepackage{lipsum} % Used for inserting dummy 'Lorem ipsum' text into the template

\linespread{1.2} % Line spacing

%\setlength\parindent{0pt} % Uncomment to remove all indentation from paragraphs

\graphicspath{{Pictures/}} % Specifies the directory where pictures are stored

\begin{document}

%----------------------------------------------------------------------------------------
%	TITLE PAGE
%----------------------------------------------------------------------------------------

\begin{titlepage}

\newcommand{\HRule}{\rule{\linewidth}{0.5mm}} % Defines a new command for the horizontal lines, change thickness here

\center % Center everything on the page

\textsc{\LARGE Politecnico di Milano}\\[1.5cm] % Name of your university/college
\textsc{\large Computer ethics}\\[0.5cm] % Minor heading such as course title

\HRule \\[0.4cm]
{ \huge \bfseries A case for bioprinting organs}\\[0.4cm] % Title of your document
\HRule \\[1.5cm]

\begin{minipage}{0.4\textwidth}
\begin{flushleft} \large
\emph{Author:}\\
Mirjam \textsc{Škarica} % Your name
\end{flushleft}
\end{minipage}
~
\begin{minipage}{0.4\textwidth}
\begin{flushright} \large
\emph{Supervisor:} \\
Prof.ssa Viola \textsc{Schiaffonati} % Supervisor's Name
\end{flushright}
\end{minipage}\\[4cm]

{\large \today}\\[3cm] % Date, change the \today to a set date if you want to be precise

%\includegraphics{Logo}\\[1cm] % Include a department/university logo - this will require the graphic package

\vfill % Fill the rest of the page with whitespace

\end{titlepage}

%----------------------------------------------------------------------------------------
%	TABLE OF CONTENTS
%----------------------------------------------------------------------------------------

\tableofcontents % Include a table of contents

\newpage % Begins the essay on a new page instead of on the same page as the table of contents 


%----------------------------------------------------------------------------------------
%	INTRODUCTION
%----------------------------------------------------------------------------------------

\section{Introduction} % Major section

Three dimensional (3D) printing is an additive manufacturing process for making a physical object from a three dimensional digital model. Meaning, it builds up the solid three dimensional objects by laying down many thin successive layers of a material. It's one of the fastest growing fields and on top of it, it is the driving force behind innovations in many areas such as engineering, manufacturing and medicine. Even though 3D printing has been around since the 1980s, 3D printing of biomaterials has become possible only recently and it has catalyzed the field of bioprinting said to revolutionize everything from the pharmaceutical to the health care industry. Bioprinting creates a complex three dimensional functional and viable tissue by layering living cells onto a biologically compatible scaffolding. This way living human tissues can be generated and matured into numerous different types of human tissue and organs including skin, kidney, liver and cartilage, to name a few. And the reasons are plentiful, ranging from scientific research and drug discovery all the way to transplants and regeneration i.e. growing organs or structures to heal and promote growth in the body. Bioprinting is a much more intricate process than non-biological 3D printing because it involves dealing with live cells. Meaning, issues arise in choosing the type of cells, the means of obtaining them, preserving them, and ways of constructing tissue. Not only do these technical issues give rise to a great number of ethical questions, but also the implications of using this rapidly evolving technology also facilitate ethical dilemmas in their own right. 

This paper will try to defend the sensitive subject of organ printing by tackling various problems regarding the potential misuse, overuse, under-use and selective use of the technology in various walks of life versus it's obvious benefits. It will first address the current ethical issues like the use of embryonic stem cells and experimentation on youth. Then the discussion will move on to potential ethical issues that are expected to arise with the development of this particular technology.

\newpage

%----------------------------------------------------------------------------------------
%	MAJOR SECTION 
%----------------------------------------------------------------------------------------

\section{Current issues} % Major section

%----------------------------------------------------------------------------------------
\subsection{The stem cell dilemma} % Sub-section

This section will address the ethical dilemma regarding the use of embryonic stem cells in the bioprinting process. 
Based on \cite{OPTN data as of January 2016} more than 121,000 men, women and children in United States alone are in need of a lifesaving organ transplant. Sadly, on average 22 people die each day waiting for a transplant. The need for efficient organ and tissue production is real and bioprinting may be a solution to this pressing issue. However, for this kind of bioprinting, embryonic stem cells are the essential part of the 3D printer's bioink.

According to a study \cite{Faulkner-Jones:2013} there is an advantage to using human embryonic stem cells in biofabrication over cells that have a more specific purpose which is that these cells have the ability to self-renew and the potential to become any other type of cell. However, their versatility is shadowed by the ethical controversy surrounding the method with which they are obtained. This is due to the fact that when the stem cells are harvested from an embryo, the extraction process actually destroys the embryo. 

The NSPE\footnote{The National Society of Professional Engineers} Code of Ethics for Engineers states as one of their fundamental canons that engineers will hold paramount the safety, health, and welfare of the public. At first, use of embryonic stem cells seems to both agree and counter this canon. 
It is in accordance with it because the research and development of this field serves the public with it's potential to save an unprecedented amount of lives. 
It seemingly contradicts the canon because it supposedly requires the taking of human life to harvest the cells and therefore, not holding public's welfare, health or safety paramount. 

However, the whole discussion about whether using embryonic stem cells is moral or not arises from a general disagreement of the status of the embryo. In other words, the disagreement about the point in time in which an embryo is considered a full human being. If we don't consider an embryo human after just 4 or 5 days following fertilization, which is the time needed for the embryo to grow to the blastocyst stage when you can harvest if for stem cells \cite{Landry:2016}, then we don't really have a moral conundrum. If, on the other hand, that is not the case, I would still like to argue that as a society we have actually already accepted the practice of utilizing embryos for stem cell research by drawing a parallel with a widely available and used reproductive technology - In Vitro Fertilization (IVF); thus rendering the controversy a bit less, well, controversial. 

In IVF, a women's eggs are extracted, inseminated in a petri dish, and after 4-5 days the now embryos are transferred into the woman's uterus. Usually more embryos are made than implanted in case the fertilization doesn't work. 
Regarding the future of those unused embryos the intended parents are left with a few choices\footnote{depending on their country's laws regarding IVF}. Usually the choices include freezing, discarding or donating the surplus embryos either to other couples trying to conceive or for scientific research. 
Indeed, current sources of hESC\footnote{human embryo stem cells} for stem cell research include excess embryos that would have been discarded as an unavoidable part of the IVF process\cite{Pediatrics:2012}.

Regarding the faith of the frozen embryos, the ones that never get used are usually eventually thawed and discarded. On the other hand, according to one study's \cite{Pavone:2011} compiled statistics on embryos that do get thawed for reuse, 86\% of those are discarded either because they didn't survive the thawing process or just because they didn't get implanted this time around. 

Furthermore, it is important to realize that in developing the IVF procedure a great number of embryos must have been used and dispensed, essentially sacrificed for science. 

Having all this in mind it is relevant to acknowledge the fact society widely accepts and uses IVF. One might go far as to say that it is considered one of the modern medical triumphs. Hence it can be concluded that by such doing, and the numbers back this claim up, regardless of whether or not we consider embryos human, we have deemed all the past and continuous "sacrifice" of embryos worthy of a greater cause. In the case of IVF the cause is creating a life.

Just like helping people conceive a child is a noble cause, so too are the efforts made not only to save people's lives by bioprinting an organ they desperately need, but also by reducing the need for human donors. And to take things to an extreme, if we could remove all together the need for adult human donors, we would end other moral and ethical dilemmas surrounding organ transplants today. The dilemma about requiring a life taken for a live given, and the issue of having to choose selectively to save one life over another. 

The development of the bioprinting field presents us with tremendous opportunities to better and to save lives. So, in my opinion, it is more wrong to stifle research with such promising medical potential over a dilemma regarding a procedure that was and in fact currently is in practice.



%----------------------------------------------------------------------------------------
\subsection{Human trials} % Sub-section

Bioprinting is a natural, albeit mind-altering, development within the 3D printing field. And the latter one has already yielded many successful surgical interventions. Just to name a few, in 2015, three infants with a life threatening airway disease were saved by 3D printing a flexible tracheal\footnotemark{} splints tailored for each individual patient \cite{Morrison:2015}.
\footnotetext{The trachea, colloquially windpipe, is a tube that connects the larynx to the bronchi of the lungs. It is an integral part of the body’s airway and has the vital function of providing air flow to and from the lungs for respiration}

Another miracle story is that of a three year old girl born with hydrocephalus\footnotemark{} who was admitted to the hospital after her brain suffered serious infections and had a high probability of rupture due to her swollen skull. She underwent an unheard of life saving surgery consisting of constructing and inserting a new 3D printed skull \cite{3dprint.com:2015}. 
\footnotetext{Hydrocephalus is a medical condition that occurs when too much cerebrospinal fluid builds up in the skull causing it to swell}

The success of these stories is enough to get one excited about the future and want to go the extra mile and start the bioprinting trials. However printing and implanting biodegradable cartilages is a far simpler endeavor than bioprinting personalized human tissue. 

One of the bigger challenges being that testing for safety is limited. For one, it is impossible to compile a reliable statistic regarding a newly printed tissue. This is due to the fact that testing the same tissue on a large number of healthy people cannot, by definition, be done because the tissue is personalized and made specially for each individual.

This raises questions about the ethicalness of offering treatment to desperate patients not fully understanding the consequences and without knowing the outcome. While this can be morally justified by offering the procedure only to people in dire need and through the process of informed consent, there might be a greater danger looming behind closed doors. 

According to \cite{Army Technology:2014} publication, The US Army is heavily investing and is hoping to soon begin clinical trials with 3D printed skin. Given the military nature of the organization, it is safe to assume the trials will be held in secrecy, better known as, confidentially. While this bit of information was made available to the public, we could infer that it is just the tip of the proverbial iceberg and that many more trials are performed by many more government bodies all over the word. If history has shown us anything it is that government agencies don't shy away from using less than ethical methods in the race to the top. Keeping the public in the dark about innovations, even if the reasons are their own safety, still leaves the people in a very vulnerable position.

Unfortunately, things have always been this way, armies have always had secret projects and a lot can be said about how actually informed is informed consent. Not to say challenging the status quo shouldn't be done, but it is to say that this is an all pervasive problem, not strictly tied to bioprinting. What is even more discouraging is the astonishingly low rate of which animal trials translate to human trials \cite{Worp:2010}. In other words, the animal models do not predict with sufficient certainty what will happen in a human body. In fact, at the time of writing, BBC News had reported an unfortunate outcome of a clinical trial in France. A new drug, a cannabis based painkiller, has left one person brain-dead and another five people in hospital, in a serious condition \cite{BBC News:2016}. Meaning that in spite of our best moral and technical efforts to minimize the risk of clinical trials, we are failing to protect people entering them.

However, there may be light at the end of the tunnel and that light is bioprinting. It is stand to usher in a new era in medicine and pharmaceutical industry. An era where testing drug therapies will be done on living human tissue instead of animals.

“What was once only possible in science fiction is moving ever closer to commercial reality,
Bioprinting can change the world for the better in terms of eradicating or mitigating the effect of disease through personalized drugs, in tissue restoration and repair, and even cosmetic surgery.

 and personalized treatments are delivered to patients.


 eradicate the risk



\newpage 

%----------------------------------------------------------------------------------------
%	MAJOR SECTION 
%----------------------------------------------------------------------------------------

\section{Future issues} % Major section

\subsection{Human enhancement i.e. healthier Frankenstein's monster} % Sub-section
When organ printing really is in full swing
http://3dprint.com/28879/3d-printing-new-unique-organs/

new meaning of body builder?
As replacement limb technology advances, people may choose to enhance their physical selves as they do with cosmetic surgery today.

superhumans
athletes
Human enhancement through bioprinting is certainly a ways off in the future, however it is a possibility that we must keep in mind as bioprinting continues to be developed. Bioprinting might make it possible to have better muscle tissue, better lungs, or a better heart. Human enhancement can range from extending the lifespan of a person, to increasing quality of life, to even creating super humans. There are many questions to be asked here, such as what if any, is the ethical amount of time to keep a person alive by replacing their body with bioprinted parts? In what ways and how much will bioprinted features change our lifestyles and appearances? Is it ethical to create a superior human being with bioprinting? This last question in particular has implications in sports and the armed forces. For example, human enhancement would almost certainly need to be banned for professional athletes, but would a medical organ transplant for an athlete be considered human enhancement?



%------------------------------------------------

\subsection{Safety} % Sub-section
open source, or just source

%------------------------------------------------

\subsection{Justice and access} % Sub-section
selective organs transplants,
wage gap,

%------------------------------------------------

\subsection{Fillet-o-print} % Sub-section
Bioprinting food
\newpage 

%----------------------------------------------------------------------------------------
%	CONCLUSION
%----------------------------------------------------------------------------------------

\section{Conclusion} % Major section

Bioprinting truly is a modern wonder but, at the risk of sounding cliché, with great power comes great responsibility. If history has tough us anything, it's that stifling innovation never works in the long run. Therefore the best thing we can do is to think long and hard about the problem, or better yet - the opportunity, in order to prepare ourselves by raising awareness, coming up with best practices, tweaking our laws, putting in place an infrastructure to better accommodate this imminent change.

\newpage

%----------------------------------------------------------------------------------------
%	BIBLIOGRAPHY
%----------------------------------------------------------------------------------------

\begin{thebibliography}{99} % Bibliography - this is intentionally simple in this template
\begin{small}
%------------------------------------------------------------------
\bibitem[Murphy and Atala (2014)]{MurphyAtala:2014}
\newblock S. V. Murphy and A. Atala (2014).
\newline {3D bioprinting of tissues and organs}
\newblock {\em Nat Biotech} 

%------------------------------------------------------------------
\bibitem[OPTN data as of January 2016]{OPTN data as of January 2016}
\newblock The Organ Procurement and Transplantation Network.
\newblock http://optn.transplant.hrsa.gov

%------------------------------------------------------------------
\bibitem[Faulkner-Jones et al. (2013)]{Faulkner-Jones:2013}
\newblock Alan Faulkner-Jones and Sebastian Greenhough and Jason A King and John Gardner and Aidan Courtney and Wenmiao Shu (2013).
\newline Development of a valve-based cell printer for the formation of human embryonic stem cell spheroid aggregates
\newblock {\em Biofabrication}

%------------------------------------------------------------------
\bibitem[Landry et al. (2016)]{Landry:2016}
\newblock Landry, Donald W., and Howard A. Zucker. (2016).
\newline Embryonic Death and the Creation of Human Embryonic Stem Cells
\newblock {\em Journal of Clinical Investigation}

%------------------------------------------------------------------
\bibitem[Pavone et al. (2011)]{Pavone:2011}
\newblock Pavone, M. E., Innes, J., Hirshfeld-Cytron, J., Kazer, R., and Zhang, J. (2011).
\newline Comparing thaw survival, implantation and live birth rates from cryopreserved zygotes, embryos and blastocysts. 
\newblock {\em Journal of Human Reproductive Sciences}
\newblock http://doi.org/10.4103/0974-1208.82356

%------------------------------------------------------------------
\bibitem[Pediatrics (2012)]{Pediatrics:2012}
\newblock COMMITTEE FOR PEDIATRIC RESEARCH and COMMITTEE ON BIOETHICS (2012).
\newline Human Embryonic Stem Cell (hESC) and Human Embryo Research
\newblock {\em Pediatrics}
\newblock http://pediatrics.aappublications.org/content/130/5/972

%------------------------------------------------------------------
\bibitem[Morrison et al. (2015)]{Morrison:2015}
\newblock Morrison, Robert J. and Hollister, Scott J. and Niedner, Matthew F. and Mahani, Maryam Ghadimi and Park, Albert H. and Mehta, Deepak K. and Ohye, Richard G. and Green, Glenn E. (2015).
\newline Mitigation of tracheobronchomalacia with 3D-printed personalized medical devices in pediatric patients
\newblock {\em Science Translational Medicine}

%------------------------------------------------------------------
\bibitem[3dprint.com (2015)]{3dprint.com:2015}
\newblock http://3dprint.com/81815/3d-printed-skull/

%------------------------------------------------------------------
\bibitem[Army Technology (2014)]{Army Technology:2014}
\newblock Dan Lafontaine, Rdecom public affairs (2014).
\newline Medical Applications for 3-D. Army invests in 3-D bioprinting to treat injured Soldiers
\newblock {\em Army Technology publication}
\newblock http://www.army.mil/e2/c/downloads/353394.pdf

%------------------------------------------------------------------
\bibitem[Worp et al. (2010)]{Worp:2010}
\newblock van der Worp, H. Bart AND Howells, David W. AND Sena, Emily S. AND Porritt, Michelle J. AND Rewell, Sarah AND O'Collins, Victoria AND Macleod, Malcolm R
(2010). 
\newline Can Animal Models of Disease Reliably Inform Human Studies?
\newblock {\em PLoS Med}
\newblock http://journals.plos.org/plosmedicine/article?id=10.1371/journal.pmed.1000245

%------------------------------------------------------------------
\bibitem[BBC News (2016)]{BBC News:2016}
\newblock http://www.bbc.com/news/world-europe-35320895

%------------------------------------------------------------------

\end{small}
\end{thebibliography}

%----------------------------------------------------------------------------------------

\end{document}
